\section{随机数的生成}
所有涉及委员会选举,权益证明的区块链都离不开随机数的应用。而区块链上的随机数与人们传统理解又有所不同:区块链本质是实现一个状态机复制的问题,故要求所有的节点以相同条件生成随机数都会得到同样的结果,这就杜绝了以物理方式生成随机数的可能(包括宇宙辐射,random.org,等等)。

区块链上的随机数基本要求是不可预测性与可验证性,否则不能满足区块链的需求:首先,出块权这种不能被预测,否则会引发一系列问题(DDOS攻击)。同时,区块链一切算法是公开的,不存在一个中心节点秘密生成随机数。这就要求所有人都能验证随机数的合法性。	

此章节以Randao白皮书为参考\footnote{https://www.randao.org/whitepaper/Randao\_v0.85.pdf},指出随机数(或随机种子)生成需要权衡的几个问题:

\textbf{最后演员问题 “last actor problem”}

当随机数的生成需要多人合作时,最后一个做出行动的成员在其他成员行动后可以预知随机数的值,而其余成员因缺少最后行动者的行动无法获知,造成信息不对称。而一旦最后行动者发现随机数对自己不利,他可以选择拒绝行动。

Algorand生成随机种子的方法为,以上一轮出块者的VRF函数(可验证)作为下一轮的随机种子。因为Algorand算法无法预测出块者,故能满足不可预测性。

但我们认为Algorand中的算法存在最后演员问题:出块权是根据账户优先级决定的,而优先级来源于每个出块者的VRF,需要进行广播。而一旦一个矿工已经接受到所有其他矿工的广播,然后发现自己的优先级是最高的,他可以提前知道自己是出块者(只要他在限定时间内广播并且所有节点正常运作)并知道下一轮随机种子。进而他也可以选择不出块来改变区块链的结果(outcome)。

Dfinity和randao采用BLS($(t,n)$门限签名)方式生成随机数。因为在得到$t$个独立签名之前无法恢复出组签名,故所有用户均无法预测组签名值,满足不可预测性。但是也存在最后演员问题:当一个用户收到$t-1$个独立签名之后,他可以提前恢复出组签名,获取随机种子,进而也可以根据结果选择拒绝运作。当然,只要剩余$n-t+1$个人不都怎么做,组签名还是能成功运作。所以,该算法能抵御非法成员少于$n-t+1$的最后演员问题,且$t$越大越难抵御。

但是,如果$t$太小的另一个明显的问题是,任何$t$个成员可以共谋,无视剩余$n-t$个成员进行所有组签名操作。故BLS能抵抗$\min\{ n-t,t-1\}$个非法成员。所以一般选取$t=(n-1)/2$。

同时,BLS的另一个问题是生成的随机种子随机性不能满足。所谓随机性指任何一个对手无法在多项式时间内区分算法返回的随机数和一个真正的随机数。Algorand的VRF签名能满足随机性要求,然而门限签名由于限制颇多,并不能理论保证返回结果的随机性。Randao采取的方式是选取多个组串行进行签名,即,前一组的输出结果作为下一组的输入进行门限签名,但仍没有理论证明。

	

