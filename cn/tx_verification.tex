% !TEX root = main.tex
\section{交易验证}
基于第\ref{sec:leader_election}章,在已选举出提案者的情况下,交易验证对出块进行验证。首先给出拜占庭环境下交易验证的相关定义和特性。

\begin{definition}
(交易验证) 对于验证者集合$V$\footnote{类似于定义\ref{def:leader_election},此处$V$可为有限或无限集合。},包含$n$个节点,其中最多有$f$个拜占庭节点,即最少有$n-f$个诚实节点。所有节点必须从提案中最终做出决策,并且满足下述条件:
\begin{itemize}
	\item 一致性(Agreement):所有诚实节点的决策必定相同。
	\item 可终止性(Termination):所有诚实节点在有限的时间内结束决策过程。
	\item 有效性(Validity):选择出的决策值必须来自某个有效的提案。
\end{itemize}
\end{definition}

对于任何安全的交易验证协议,都需要满足上述条件\footnote{部分研究中将可终止性和有效性描述为活性(Liveness)。}。如\ref{subsec:intro_tx_verification}小节介绍,交易确认机制根据块是否具有终结性划分为链式协议和基于投票协议。投票式交易验证满足终结性,即交易一旦完成验证则不可能发生改变。对应地,链式交易验证则仅满足概率终结性,即随着链结构的增长交易发生改变逐渐降低,但无法为$0$。需要指出的是,即便是类似PBFT的投票协议,其数据结构仍可以采用链式区块结构。

\subsection{投票协议}
从安全性角度,我们不希望已经验证的交易会存在修改的可能,即出现长距离攻击(Long-Range Attack)。同时,随着数据的不断增长,一些普通节点可能无法负担庞大的数据量,因此确定性的终结性可以减少数据的存储。最后,考虑到未来的分片设计,数据分片需要终结性作为基础。因此我们采用基于投票机制的设计,保证交易的终结性。

对于投票协议,除了上述特性之外,我们认为其需要考虑如下因素:
\begin{itemize}
	\item 投票效益:通常而言区块链共识的投票是根据参与者的持有(或抵押)的资产计算所得,在没有投票成本的情况下,验证者更倾向分散其投票效益。一种解决方法是引入投票成本,或者对于分散投票行为作出惩罚\cite{buterin2017casper}。此外,投票效益的计算必须能够抵抗女巫攻击(Sybil Attack)。
	\item 委员会选举:为防止贿选以及共谋现象(Colluding),投票前投票节点身份不应曝光,同时在每次投票后(例如公布对某个区块的签名公证),其将丧失委员会资格,直至再次入选委员会。
	\item 延迟:尽管任何共识机制都需要保证在有限时间内结束决策,但我们仍希望投票能够快速达成一致,从而减少交易的确认延时(confirmation delay)。
\end{itemize}

