% !TEX root = main.tex
\section{一致达成}
\label{sec:tx_verification}
\subsection{基础}
基于第\ref{sec:leader_election}章,在出块节点提出区块的情况下,验证节点\footnote{部分系统对出块和验证节点不作区分。}需要对区块的顺序达成一致,该问题可以抽象为分布式系统中的状态机复制(State Machine Replication)问题\cite{pass2017rethinking}。

%首先给出拜占庭环境下交易验证的相关定义和特性。


%对于节点集合$V$\footnote{类似于定义\ref{def:leader_election},此处$V$可为有限或无限集合。},
首先我们给出区块链中拜占庭容错的状态机复制问题定义。
\begin{definition}
(拜占庭容错状态机复制)对于验证节点集合$V$,其中最多有$f$个拜占庭节点(即最少有$n_v-f$个诚实节点),所有验证节点必须从提案中最终做出决策,并且满足下述条件:
\begin{itemize}
	\item 安全性(Safety)\footnote{部分研究描述为一致性(Consistency)。}:所有诚实(验证)节点对所有提议区块达成顺序一致性;
	\item 活性(Liveness)\footnote{部分研究描述为可终止性(Termination)和有效性(Validity)。}:所有诚实(出块)节点提出的区块最终都会被记录;
	%\item 可终止性(Termination):所有诚实节点在有限的时间内结束决策过程。
	%\item 有效性(Validity)\footnote{部分研究中将可终止性和有效性描述为活性(Liveness)。}:选择出的决策值必须来自某个有效的提案。
\end{itemize}
\end{definition}

根据FLP定律,在完全异步网络拜占庭环境下,不存在确定性的算法满足上述条件\cite{fischer1982impossibility}。因此现有区块链交易验证算法都是同步(或者所谓的半同步)网络拜占庭环境下的状态机复制算法\cite{gilad2017algorand,castro1999practical}。更进一步地,在有拜占庭节点的情况下,现有的区块链共识算法在异步网络中应该优先保证安全性而舍弃活性\footnote{区块链共识算法的发展趋势,https://zhuanlan.zhihu.com/p/52251671}。

%保证异步网络的,同步网络的活性

\subsection{性质分析}
除了上述的特性之外,我们需要考虑另一个重要性质,最终性(Finality)。

在区块链系统中,由于网络延迟以及恶意出块因此在同一高度可能会出现多个区块即分叉。因此所有验证节点需要对分叉进行协定,即选择某个区块从而达到共享状态。最终性是指由验证节点决策通过的共享状态不会被改变,最终性又被称为绝对最终性(Absolute Finality),对应地,存在一定概率使已通过区块状态改变称之为概率最终性(Probabilistic Finality)\footnote{Finality in Blockchain Consensus, https://medium.com/mechanism-labs/finality-in-blockchain-consensus-d1f83c120a9a}。

根据是否具有最终性,现有的共识算法又可以划分为链式协议和基于投票协议\cite{wang2019survey}。

\subsubsection{链式协议}
在采用链式协议的系统中,通常并不区分出块节点和验证节点。诚实的出块节点会遵从某种规则(最长链原则或者最重子树原则)产生区块,当出现分叉后,根据上述规则所有节点达成一致共享状态。

首先,链式协议仅能保证最终一致性(Eventual Consistency)\footnote{最终一致性是弱一致性(Weak Consistency)的一种形式},即如果不再有新的状态更新,所有节点最终就共享状态达成一致,但在某个时间段内,共享状态在各个节点所存的状态可能不一致\cite{brewer2000towards}。

同时,链式协议仅满足概率最终性,即随着链结构的增长区块状态改变的概率逐渐降低,但永远无法达到零,即仍然存在被改变可能。比特币白皮书给出了概率最终性的简单分析,基于最长链原则,当恶意节点产生区块的概率$q=0.1$时,恶意节点从6个区块高度前开始重新构造子链从而逆转当前主链的概率低于$0.0003$\cite{nakamoto2008bitcoin},因此比特币网络的交易确认时间通常约定俗成为一小时即6个区块高度。

随之而来的是,采用链式协议的系统需要等待一段时间才能提供足够的概率最终性,因此往往链式协议系统的延迟(Latency)都较高\cite{bano2017consensus}。

从安全角度来看,概率最终性意味着已经验证的交易会存在修改的可能,例如出现双重支付(Double-Spending)或者长距离攻击(Long-Range Attack)等\cite{garay2015bitcoin,bahack2013theoretical},更详细的分析见附录\ref{ch:attack}。

%虽然比特币等采用链式验证协议的系统认为随着最长链的增长其被扭转概率会非常低,但实际上目前比特币网络中排名前四的矿池已经占据了全网$50\%$以上算力\footnote{https://www.blockchain.com/pools.},这意味着其安全性是由矿池决定而并非系统本身。因此理论上安全的区块链系统必须满足最终性。


\subsubsection{投票协议}
%投票式交易验证满足最终性,即交易一旦完成验证则不可能发生改变。对应地,需要指出的是,即便是基于投票的验证协议,其底层数据结构仍可以采用链式区块结构。
%同时,随着数据的不断增长,一些普通节点可能无法负担庞大的数据量,因此最终性可以减少数据的存储。最后,考虑到未来的分片设计,数据分片需要最终性作为基础。
%综上,我们采用基于投票机制的设计,保证交易验证的最终性。
为了实现最终性,许多区块链系统采用了传统分布式系统的共识算法。论文\cite{howard2019consensus}指出现有的分布式系统共识大多基于经典的Paxos模型\cite{lamport2001paxos},在该模型下,所有接受节点通过投票对提案者的提案进行表决,此类协议被称为投票协议。

%投票协议满足最终性,因此交易一旦完成验证则不可能发生改变。
%对应地,需要指出的是,即便是基于投票的验证协议,其底层数据结构仍可以采用链式区块结构。
%\subsection{已有投票策略分析}
%首先我们对目前基于投票的验证策略进行简单分析。
%根据是否需要验证节点进行身份验证,我们将投票策略分为授权投票和无授权投票,通常来说分别适用于联盟链和公链。

对于联盟链为代表的系统\cite{androulaki2018hyperledger},通常采用PBFT\cite{castro1999practical}和投机BFT算法\cite{kotla2007zyzzyva}。
%\subsubsection{授权投票策略:PBFT}
%协议中负责提案的节点称之为主节点(Primary),其余所有的接受节点称之为副本节点(Backups)。在已经指定领袖节点的情况下,验证过程如下:
%\begin{itemize}
%	\item PRE-PREPARE:主节点在收到请求后向所有副本节点广播与准备消息,其格式为$\langle \langle PRE-PREPARE,v,n,d \rangle_{\sigma p},m \rangle$,其中$v$是视图编号,$n$是消息序号,$d$是消息摘要,$m$为请求消息,$\sigma _p$为主节点$p$的签名;
%	\item PREPARE:一旦副本节点$i$接受预准备消息则进入准备阶段,同时该节点向所有副本节点发送准备消息$\langle PREPARE,v,n,d,i\rangle_{\sigma i}$。当节点$i$收到$2f$个从不同副本节点发来与$PRE-PREPARE$相匹配的$PREPARE$消息,则定义$prepared(m,v,n,i)$为真;
%	\item COMMIT:当$prepared(m,v,n,i)$为真时,副本节点$i$将$\langle COMMIT,v,n,D(m),i\rangle$广播至其他副本节点,进入确认阶段。对节点$i$而言,$prepared(m,v,n,i)$为真且$i$已经接受了$2f+1$个$COMMIT$消息与$PRE-PREPARE$消息一致则定义$committed-local(m,v,n,i)$为真。而存在$f+1$个正常副本节点集合使得其中所有副本节点$i$的$prepared(m,v,n,i)$为真,则定义$committed(m,v,n)$为真。
%\end{itemize}
%\textcolor{gray}{预准备阶段和准备阶段确保所有正常节点对同一个视图中的请求序号达成一致。而确认阶段保证了所有正常阶段对本地确认的请求序号达成一致,及时这些请求在每个节点的确认处于不同的视图。}
在PBFT策略中,验证节点为固定的集合,并且所有节点身份公开,在其发布PREPARE消息后其决策也公开可见,因而在其发布正常COMMIT消息前存在被攻击或者贿赂可能。同时,此类BFT算法在一般情况下网络通信开销为$O(n^2)$,因此验证节点的规模通常较小。综上对于采用PBFT算法的系统,无法适用于无授权环境。

%\subsubsection{无授权投票:Algorand}
%无授权场景下节点会动态加入离开,同时由于节点没有身份认证,
为适用于更大的节点规模,Byzcoin从出块节点中选择验证节点\cite{kogias2016enhancing},Algorand采用抽签方式选择验证节点\cite{gilad2017algorand},本质上来看,两者都通过随机方式(PoW或者伪随机数)从一个大规模集合中选择更小规模的验证集合用于投票。同时,与PBFT等传统分布式共识算法不同的是,Algorand采用的投票算法$BA\bigstar$并未区分主节点和备份节点,在整个投票过程中,每个验证节点都需要多次执行$CommitteeVote()$和$CountVotes()$操作,即进行多次的投票和计票。Algorand假设在整个$BA\bigstar$过程中所有验证节点不会改变投票选择,在保证一定比例诚实节点的情况下,如果收不到足够的投票则问题一定出在网络上\footnote{Algorand假设网络最终会收敛于同步模型}。%但由于诚实节点仍可能会被攻击或者贿赂,我们认为该算法仍然存在问题。

关于投票的研究有很多,论文\cite{kiayias2002self}指出电子投票系统需要满足若干特性,详细分析见附录\ref{ch:voting}。

%不同于PBFT,Algorand\cite{gilad2017algorand}属于无授权场景共识 协议,对应地,Algorand在领袖选举和交易验证中均采用了基于VRF的随机抽取策略。这里重点分析其交易验证部分。

%Algorand验证的核心算法$BA\bigstar$来自于\cite{miller2016honey},具体如下:
%\begin{itemize}
%	\item Reduction:该步骤的目标是将需要达成共识的多个区块转化为对某一特定区块或者空块二选一达成共识;
%	\item BinaryBA$\bigstar$:基于Reduction的输出,在特定区块和空块之间达成共识。在$Maxstep$轮中没有达成共识则认为网络状况出现问题。
%\end{itemize}

%需要注意的是,在Reduction和BinaryBA$\bigstar$中,都需要多次执行$CommitteeVote()$和$CountVotes()$操作,即投票和计票。这里重点分析前者,$CommitteeVote()$输入包括$(ctx,round,step,\tau,value)$,其中$ctx$为环境变量(包括当前账本信息以及当前种子等等),$round$表示当前轮数,$step$表示执行当前计票操作的步骤(例如Reduction-1),$\tau$表示抽签比例参数,$value$表示投票区块。

%验证节点执行$CommitteeVote()$时需要先执行$Sortition()$即通过抽签判断自己是否有资格参与投票,不难发现,在给定$round$和随机种子($ctx.seed$)情况下,每个验证节点每次执行$Sortition()$的结果固定。因此实际上在Reduction和BinaryBA$\bigstar$过程中,仍然是同一批节点在参与投票,并且在Reduction中的第一次$CommitteeVote()$之后,所有投票的节点身份已经曝光,因此也存在被攻击或者贿赂的风险。

%Algorand假设在整个$BA\bigstar$过程中所有验证节点不会改变投票选择,在保证三分之二诚实节点的情况下,如果收不到足够的投票则问题一定出在网络上\footnote{Algorand假设网络最终会收敛于同步模型}。但由于诚实节点仍可能会被攻击或者贿赂,我们认为该算法仍然存在问题。


%\subsection{投票模型}


%即便,其仍可能通过共谋导致潜在的安全问题。
%我们认为理想的投票机制应该抵抗上述三种恶意行为。首先,我们给出投票流程中的相关定义:

%\begin{definition}
%	(注册)对于任何期望参与验证过程的节点$i$,执行操作$R(s_i,\varepsilon)$,$s_i$为节点$i$的状态,$\varepsilon$为额外证明输入\textcolor{gray}{(通常为某个随机数)},$R$输出为
%\end{definition}

%\begin{definition}
%(投票) 对于验证节点$i$,在接受某提案区块$B$后,满足$V(B,t)=1$的情况下,对所有验证节点广播消息$\langle pk_i,sign_{sk_i}(h(B),\pi_i) \rangle$。其中$sign_{sk_i}(\cdot)$表示基于节点$i$私钥的签名,$h(B)$表示提案区块$B$的哈希值,$\pi_i$表示验证节点$i$的投票效益证明。
%\end{definition}

%\begin{definition}
%	(验票)
%\end{definition}

%\begin{definition}
%(计票) 对于验证节点$j$,对所有验证节点广播消息$\langle pk_i,sign_{sk_i}(h(B),\pi_i) \rangle$。其中$sign_{sk_i}(\cdot)$表示基于节点$i$私钥的签名,$h(B)$表示提案区块$B$的哈希值,$\pi_i$表示验证节点$i$的投票效益证明。
%\end{definition}


此外,对于共识中的投票协议,其投票效益的计算也不应被忽视。许多共识系统采用的是“一人一票”制度,即任何有投票权的验证节点只能投一票,但验证节点获取投票权的概率和其持有(或抵押)的资产相关\cite{gilad2017algorand,buchman2016tendermint},这可能导致系统容易收到女巫攻击(Sybil Attack)\cite{douceur2002sybil}。在非授权的投票系统中,攻击者可以通过创建大量假名节点来增加其获取投票权的概率。因此投票过程必须能够抵抗女巫攻击,解决方法可以是通过设置准入门槛\cite{david2018ouroboros},或者采用类似于星云指数的计算方式\cite{Nebulasyellowpaper}。

%投票协议也需要满足如下特性:

%部分共识采用的投票协议并非“一人一票”,其票数是根据参与者的持有(或抵押)的资产计算所得\cite{gilad2017algorand},因此可能会出现如下安全问题:

%对于投票效益的计算,我们认为也可能存在如下潜在的安全问题:

%\begin{itemize}
%	\item 抗女巫攻击:传统意义上的女巫攻击(Sybil Attack)是指攻击者通过创建大量的假名表示来破坏对等网络的评价系统\cite{douceur2002sybil},在非授权的投票系统中,攻击者可以通过创建大量假名节点来增加其获取投票权的概率。因此投票过程必须能够抵抗女巫攻击,解决方法可以是通过设置准入门槛\cite{david2018ouroboros},或者采用类似于星云指数的计算方式\cite{Nebulasyellowpaper};
%	\item Nothing-at-Stake:无利害关系(Nothing-at-Stake)是指在没有投票成本的情况下,验证者更倾向分散其投票效益。一种解决方法是引入投票成本,或者对于分散投票行为作出惩罚\cite{buterin2017casper}。	
	%\item 共谋:为防止贿选以及共谋现象(Colluding),投票前投票节点身份不应曝光,同时在每次投票后(例如公布对某个区块的签名公证),其将丧失投票资格,直至再次入选委员会。
	%\item 延迟:尽管任何共识机制都需要保证在有限时间内结束决策,但我们仍希望投票能够快速达成一致,从而减少交易的确认延时(confirmation delay)。
%\end{itemize}
